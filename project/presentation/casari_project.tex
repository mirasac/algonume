%%%%%%%%% Notes %%%%%%%%%

%%%%%%%%% Packages %%%%%%%%%
\documentclass[aspectratio=43]{beamer}
\usepackage[T1]{fontenc}
\usepackage[utf8]{inputenc}
\usepackage{pgfpages}

%%%%%%%%% Packages setup %%%%%%%%%
\setbeameroption{show notes on second screen}
%\setbeameroption{show only notes}
%\setbeamerfont{note page}{size=\tiny}
\setbeamercolor{note page}{bg=white, fg=black}
\setbeamercolor{note title}{bg=white!99!black, fg=black}
\usetheme{Hannover}
\usecolortheme{spruce}
\graphicspath{{./figures/}{../report/figures/}}

%%%%%%%%% Document informations %%%%%%%%%
\title{Radiative-convective equilibrium in a grey atmosphere}
\author{Marco Casari}
\date[03/10/2023]{Complex systems in climate physics, 3 October 2023}
\institute[UniTo]{University of Turin}

%%%%%%%%% Document %%%%%%%%%
\begin{document}
\begin{frame}
  \titlepage
  \note{
    \begin{itemize}
      \item Radiative-convective models.
      \item Atmospheric temperature profile averaged over all latitudes and longitudes.
      \item Radiative and convective processes dominates.
      \item Grey atmosphere.
      \item Temperature profile is studied under radiative equilibrium and radiative-convective equilibrium.
    \end{itemize}
  }
\end{frame}

\section{Introduction}
\begin{frame}{Introduction}
  \begin{itemize}
    \item<1-> 
    % MC continue.
  \end{itemize}
\end{frame}

\begin{frame}{Hypotheses}
\end{frame}

\begin{frame}{Vertical coordinates}
\end{frame}



\section{Radiative equilibrium}
\begin{frame}{Analytical solution}
\end{frame}

\begin{frame}{Numerical solution}
\end{frame}



\section{Radiative-convective equilibrium}
\begin{frame}{Radiative-convective equilibrium}
\end{frame}



\section{Conclusion}
\begin{frame}{Conclusion}
  % MC put conclusive summary with items.
%  \vfill
%  \onslide<2->{
%    \centering
%    \huge
%    Thank you
%  }
\end{frame}
\end{document}
