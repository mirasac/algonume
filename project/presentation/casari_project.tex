%%%%%%%%% Notes %%%%%%%%%

%%%%%%%%% Packages %%%%%%%%%
\documentclass[aspectratio=43]{beamer}
\usepackage[T1]{fontenc}
\usepackage{derivative}
\usepackage{pgfpages}

%%%%%%%%% Packages setup %%%%%%%%%
\setbeameroption{show notes on second screen}
%\setbeameroption{show only notes}
%\setbeamerfont{note page}{size=\tiny}
\setbeamercolor{note page}{bg=white, fg=black}
\setbeamercolor{note title}{bg=white!99!black, fg=black}
\usetheme{Hannover}
\usecolortheme{spruce}
\graphicspath{{./figures/}{../report/figures/}}

\newcommand{\zTOA}{z_\text{TOA}}
\newcommand{\PTOA}{P_\text{TOA}}
\newcommand{\deltaTOA}{\delta_\text{TOA}}

%%%%%%%%% Document informations %%%%%%%%%
\title{Radiative-convective equilibrium in a grey atmosphere}
\author{Marco Casari}
\date[03/10/2023]{Complex systems in climate physics, 3 October 2023}
\institute[UniTo]{University of Turin}

%%%%%%%%% Document %%%%%%%%%
\begin{document}
\begin{frame}
  \titlepage
  \note{
    \begin{itemize}
      \item A radiative-convective model is used to study a grey atmosphere.
      \item Comparison between numerical and analytical solutions is possible in radiative equilibrium.
    \end{itemize}
  }
\end{frame}

\section{Introduction}
\begin{frame}{Introduction}
  \begin{itemize}
    \item<1-> Average vertical temperature profile $T(t, z)$ of atmosphere.
    \note[item]<1->{The analysed quantity is the atmospheric temperature profile averaged over all latitudes and longitudes.}
    \item<2-> Radiative Transfer Equation (RTE).
    \note[item]<2->{RTE describes radiative processes.}
    \item<3-> Fluid dynamics equations.
    \note[item]<3->{Fluid dynamics equations describe convective processes.}
  \end{itemize}
\end{frame}

\begin{frame}{Hypotheses}
  \begin{itemize}
    \item<1-> Thermodynamic energy equation in Local Thermodynamic Equilibrium (LTE):
    \begin{equation}
      \label{eq:temperature_derivation}
      \pdv{T}{t} = -\frac{1}{\varrho c_P} \pdv{q}{z}
      \quad .
    \end{equation}
    \item<2-> Radiative-convective equilibrium.
    \item<3-> Grey atmosphere.
    \note[item]<3->{Quantities do not depend on the frequency of electromagnetic radiation.}
  \end{itemize}
\end{frame}

\begin{frame}{Additional hypotheses}
  \begin{itemize}
    \item<1-> Hypotheses on the planet.
    \note[item]<1->{Diurnal cycle, constant irradiance, constant Bond albedo, blackbody surface, constant gravitational acceleration.}
    \item<2-> Hypotheses on the atmosphere composition.
    \note[item]<2->{Hydrostatic equilibrium, constant specific heat at constant pressure, scattering is neglected, absorption coefficient depends only on altitude, constant mass attenutation coefficient, ideal gas.}
    \item<3-> Hypotheses on total heat flux.
    \note[item]<3->{Heat flux determined only by radiative and convective processes, two-stream approximation, numerical correction for convection.}
  \end{itemize}
\end{frame}

\begin{frame}{Vertical coordinates}
  \begin{itemize}
    \item Relation between pressure and altitude:
      \begin{equation}
        \label{eq:pressure_altitude}
        P(z) = P_\text{g} \exp{\bigg( - \frac{z - z_\text{g}}{z_0} \bigg)}
        \quad .
      \end{equation}
    \item Relation between optical depth and pressure:
      \begin{equation}
        \label{eq:optical_depth_pressure}
        \delta(P) = \frac{\mu_\text{m}}{g} (P - \PTOA)
        \quad .
      \end{equation}
    \item Relation between optical depth and altitude:
      \begin{equation}
        \label{eq:optical_depth_altitude_2}
        \delta(z) = \frac{\mu_\text{m}}{g} \bigg( P_\text{g} \exp{\bigg( - \frac{z - z_\text{g}}{z_0} \bigg)} - \PTOA \bigg)
        \quad .
      \end{equation}
  \end{itemize}
\end{frame}



\section{Radiative equilibrium}
\begin{frame}{Analytical solution}
  \begin{itemize}
    \item RTE for non-scattering medium in LTE:
      \begin{equation}
        \label{eq:RTE_analytical}
        \frac{1}{\mu} \pdv{L}{z} = B_\nu - L
        \quad .
      \end{equation}
  \end{itemize}
\end{frame}

\begin{frame}{Numerical solution}
\end{frame}



\section{Radiative-convective equilibrium}
\begin{frame}{Radiative-convective equilibrium}
\end{frame}



\section{Conclusion}
\begin{frame}{Conclusion}
  % MC put conclusive summary with items.
%  \vfill
%  \onslide<2->{
%    \centering
%    \huge
%    Thank you
%  }
\end{frame}
\end{document}
