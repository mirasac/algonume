\documentclass[a4paper,10pt,draft]{article}

% Packages.
\usepackage[T1]{fontenc}
\usepackage{amsmath}
\usepackage{amssymb}
\usepackage[style=ieee,backend=biber]{biblatex}
%\usepackage{booktabs}
%\usepackage{caption}
\usepackage{hyperref}
\usepackage{graphicx}
\usepackage{listings}
%\usepackage{mathtools}
%\usepackage{physics}
\usepackage{siunitx}
%\usepackage{subcaption}
%\usepackage{subfiles}
%\usepackage{tabularx}
%\usepackage{varioref}
\usepackage{xcolor}

% Packages settings.
\addbibresource{bibliography.bib}
%\captionsetup{tableposition=top,figureposition=bottom,font=footnotesize,format=hang,subrefformat=parens}
\hypersetup{hidelinks}
%\DeclareGraphicsExtensions{.pdf,.jpg,.png}
\graphicspath{{./figures/}}
\sisetup{group-digits=integer,separate-uncertainty,per-mode=fraction}

% Customization.
\definecolor{Blue}{rgb}{0.2,0.2,0.9}
\definecolor{Green}{rgb}{0,0.6,0}
\definecolor{Gray}{rgb}{0.5,0.5,0.5}
\definecolor{Purple}{rgb}{0.58,0,0.82}
\definecolor{background}{rgb}{0.98,0.98,0.95}
\lstdefinestyle{mystyle}{
  backgroundcolor=\color{background},
  commentstyle=\color{Green},
  keywordstyle=\color{Blue},
  numberstyle=\tiny\color{Gray},
  stringstyle=\color{Purple},
  basicstyle=\ttfamily\footnotesize,
  breakatwhitespace=false,
  breaklines=true,
  captionpos=b,
  keepspaces=true,
  numbers=left,
  numbersep=5pt,
  showspaces=false,
  showstringspaces=false,
  showtabs=false,
  tabsize=2
}
\lstset{style=mystyle}

\newcommand{\cpp}{C++}

\raggedbottom

% Document.
\begin{document}
\title{Study Nuclear Winter with a radiative-convective climate model} % MC choose title.
\author{Marco Casari}
\date{\today}
\maketitle

\begin{abstract}
  % MC abstract.
\end{abstract}

%\tableofcontents
%\newpage

\section{Introduction}
% MC the aim of this project is presented.
% MC the physical background is presented.

%\section{Materials}

\section{Methods}
% MC explain the subdivision between the radiative part and the convective part.
% MC distinguish in appropriate subsections the treatment of shortwave and longwave radiations and radiation scattering vs absorption.

\subsection{Longwave radiation}
Scattering of solar radiation at high wavelength $\lambda \geq \SI{4}{\micro\metre}$ is negligible with respect to absorption in atmosphere, moreover radiation at these wavelenghts has lower intensities than radiation emitted by Earth's surface and atmosphere. For these reasons longwave radiation is considered to be emitted only by Earth's surface and atmosphere.\cite[468]{https://doi.org/10.1029/RG016i004p00465}

\subsection{Shortwave radiation}

\subsection{Numerical approach} % MC maybe change title.
% MC convert the problem and the physical formulae into the numerical algorithms and functions to solve it.
% MC show and describe formulae.

\section{Results}
% MC explain briefly how to read a skew-T plot.
% MC show and comment obtained plots.

\section{Discussion}
% MC sum up what I did and the results compared with article TTAPS-I.
% MC if there is enough time, use the results and the comparison with TTAPS-I as model validation then gain some insights on the same results using the contemporary nuclear arsenal and compare with the modern version of TTAPS (i.e. Turco 2008, compare what is possible since it is made with a GCM).

%\section{Conclusion} % MC instead create a Discussion and conclusion section.S

\newpage
\appendix

\section{Source code}
In this section the \cpp\ code used to obtain the results presented in this work is shown and commented.
% MC show the whole source code, splitted in sections.

\subsection{Classes}
% MC show and comment the source code for the used classes.

%\newpage % MC probably not needed.
\printbibliography[heading=bibintoc]
\end{document}
